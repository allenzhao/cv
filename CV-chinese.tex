%# -*- coding:utf-8 -*-
%% start of file `template-zh.tex'.
%% Copyright 2006-2012 Xavier Danaux (xdanaux@gmail.com).
%
% This work may be distributed and/or modified under the
% conditions of the LaTeX Project Public License version 1.3c,
% available at http://www.latex-project.org/lppl/.
      
      
\documentclass[12pt,a4paper,sans]{moderncv}   % possible options include font size ('10pt', '11pt' and '12pt'), paper size ('a4paper', 'letterpaper', 'a5paper', 'legalpaper', 'executivepaper' and 'landscape') and font family ('sans' and 'roman')
      
% moderncv 主题
\moderncvstyle{classic}                        % 选项参数是 ‘casual’, ‘classic’, ‘oldstyle’ 和 ’banking’
\moderncvcolor{orange}                          % 选项参数是 ‘blue’ (默认)、‘orange’、‘green’、‘red’、‘purple’ 和 ‘grey’
%\nopagenumbers{}                             % 消除注释以取消自动页码生成功能
      
% 字符编码
\usepackage{fontspec,xunicode}
      
% 设置字体
\usepackage{xeCJK}                            % 使用xelatex的CJK包
\xeCJKsetup{
    AutoFakeBold=true,                        % AutoFakeBold在未设置粗体时使用伪粗体
    AutoFakeSlant=false,                       % AutoFakeSlant在未设置斜体时使用伪斜体
}
\setCJKmainfont{WenQuanYi Micro Hei}          % 设置正文罗马族的 CJK 字体
      
% 调整页面出血
\usepackage[scale=0.92]{geometry}
%\setlength{\hintscolumnwidth}{3cm}           % 如果你希望改变日期栏的宽度
      
% 个人信息
\firstname{赵}
\familyname{泽涵}
\address{10029 北京市朝阳区惠新东街10号}{对外经济贸易大学汇智公寓457室}             % 可选项、如不需要可删除本行
\mobile{+86~185~0195~9574}                         % 可选项、如不需要可删除本行
\email{cnallenzhao@gmail.com}                    % 可选项、如不需要可删除本行
%\photo[60pt][0pt]{allenzhao}    
\social[github]{allenzhao} 
\social[linkedin]{zehanzhao}   
%\github{allenzhao}
\title{个人简历}              % ‘64pt’是图片必须压缩至的高度、‘0.4pt‘是图片边框的宽度 (如不需要可调节至0pt)、’picture‘ 是图片文件的名字;可选项、如不需要可删除本行
      
% 显示索引号;仅用于在简历中使用了引言
%\makeatletter
%\renewcommand*{\bibliographyitemlabel}{\@biblabel{\arabic{enumiv}}}
%\makeatother
      
% 分类索引
%\usepackage{multibib}
%\newcites{book,misc}{{Books},{Others}}
%----------------------------------------------------------------------------------
%            内容
%----------------------------------------------------------------------------------
\usepackage[unicode,pdfencoding=auto]{hyperref} % 避免由hyperref引起的警告
\begin{document}
\maketitle
\renewcommand{\listitemsymbol}{-}             % 改变列表符号
      
\section{教育背景}
\cventry{2011年-今}{管理学学士}{信息学院电子商务专业}{对外经济贸易大学}{北京}{
核心课:宏微观经济学, 程序设计 (C), 数据库系统, 软件开发工具, 面向对象程序设计 (Java), 电子商务 系统建设与实施, 数据结构, 计算机网络, 高级商务英语等}
\cventry{2015.9}{管理科学与工程硕士}{经济管理学院}{清华大学}{北京}{}
\section{IT技能及英语能力}
\cvlistitem{~1年PHP开发经验;半年Ruby on Rails 开发经验。会写 HTML/CSS,常用 JS(jQuery),熟悉Git(GitHub)/SVN,日常使用 OS X 进行开发。}
\cvlistitem{ ~英语水平:流利地读写,听力及口语能力良好,交流无障碍。CET4:616,CET6:602。}
\section{自我评价}
\cvitem{}{性格积极乐观,做事认真,责任心强,学习能力强,能够较快的适应新环境、学习新事物,热爱团队合作,有创造力,乐于迎接挑战,并且坚持不懈。}

\section{工作经验与项目经历}

\cventry{2015年1月 - 6月}{实习全栈工程师}{华兴资本}{北京}{}{
主要职责:
\begin{itemize}
\item 在Tech组负责协助设计并实现面向内部及外部的平台
\item 使用Ruby on Rails完成开发,编写测试
\item 在工作中实践多种敏捷方法,如Scrum, User Story等
\end{itemize}}

\cventry{2014年5月 - 8月}{2014 谷歌编程之夏学生 (GSoC 2014 Student)}{Joomla 开源基金会}{远程工作}{}{
主要职责:
\begin{itemize}
\item 入选 Google SoC Student(8 人,共 34 人申请),为 Joomla 开源基金会旗下的 Issue Tracker 贡献代码
\item 设计并实现 AJAX 读取列表,设计并实现 Issue 分类功能
\item 帮助 Joomla Issue Tracker 完成本地化。与其他开发者协作,预计项目结束后发布 1.0 版本
\end{itemize}}

\cventry{2014年1月 - 6月}{PHP 开发工程师(实习)}{腾讯网儿童频道}{北京}{}{
主要职责:
\begin{itemize}
\item 开发和维护频道某产品,协助产品经理完成需求分析
\item 独立开发管理平台新功能,设计复杂数据库结构,帮助维护管理平台,应对用户量增长
\item 开发并重构与 Flash 交互的 API,提高响应速度。在职期间,用户量从 420 万增长到 530 万
\end{itemize}}

\section{获奖情况}
\cvlistitem{2013年华北五省计算机应用大赛本科组三等奖}
\cvlistitem{2013经纬杯创业比赛优胜奖}
      
      
% 来自BibTeX文件但不使用multibib包的出版物
%\renewcommand*{\bibliographyitemlabel}{\@biblabel{\arabic{enumiv}}}% BibTeX的数字标签
% 'publications' 是BibTeX文件的文件名
      
% 来自BibTeX文件并使用multibib包的出版物
%\section{出版物}
%\nocitebook{book1,book2}
%\bibliographystylebook{plain}
%\bibliographybook{publications}               % 'publications' 是BibTeX文件的文件名
%\nocitemisc{misc1,misc2,misc3}
%\bibliographystylemisc{plain}
%\bibliographymisc{publications}               % 'publications' 是BibTeX文件的文件名
      
\end{document}
      
      
%% 文件结尾 `template-zh.tex'.